\documentclass[../main.tex]{subfiles}


\begin{document}
	\section{Introduction}
		By analysing weather data we can obtain a lot of information about our climate, changes to it and fluctuations from year to year. This project will use ROOT from Cern to analyse data from a number of weather stations positioned in different parts of Sweden. Using this data we pick a date and measure the temperature for that specific date over a few years in order to see the fluctuations over the years. Next we looked at the temperature of all days over a year. We could here see that the temperature starts low in the beginning of the year, increases towards the middle of the year and the decreases again towards the end as expected. We also search for the warmest and coldest days of every year. This gives a fairly wide spread of days over the years where the warmest and coldest days occur. Finally we looked at the highest temperature of a year for all recorded datasets. By looking at this graph we can see that the highest temperatures climb and fall together. 
\end{document}