\documentclass[../main.tex]{subfiles}

\begin{document}

\section{Conclusion}

By analyzing data, a great deal of hidden information can be obtained. This could be 
seen in this project, where a huge amount of temperatures were analyzed. Once the 
first challenge of the project - and perhaps the most important one - was passed and the 
data could be read, the actual analysis could begin. \\

\noindent First two histograms were made, with 
the datasets from Luleå and Lund, of March 3rd. From these histograms, the mean 
temperatures of that day in the two cities were found to be $-8.37^\circ$C for Luleå and 
$0.44^\circ$C for Lund. The average temperature of each day, also in Lund, was then calculated.
The plotted results showed that the temperature starts low at the beginning of the year, 
increases towards the middle of the year and then decreases again towards the end of it. 
Thereafter, the warmest and coldest days in Lund 
were searched for. A histogram was made and afterwards fitted with a gaussian. The mean
for the warmest day was found to be 196 and for the coldest it was 24, meaning that the 
warmest day in Lund is most likely to occur in June, while the coldest is most likely 
to occur in January. 
In the final part of this project, the highest temperature of the year for all cities, 
was obtained. In the results all cities were plotted in the same (messy) graph. From this
graph it can be seen that the highest temperatures climb and fall together. \\ 


\noindent Although the results obtained in this project are all quite expected and not very 
surprising, it is always good to be able to confirm ideas and perceptions. Moreover, finding 
out what kind of information can be extracted with the right analysis tools and what the many
possibilities for this are, is fascinating. 

\end{document}

